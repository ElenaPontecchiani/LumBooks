Per la validazione dei file HTML sono stati utilizzati due servizi: TotalValidator\footnote{https://www.totalvalidator.com} e il validatore di W3C\footnote{https://validator.w3.org} . Per il CSS del sito è stato sfruttato un altro validatore sempre offerto dal W3C\footnote{http://www.css-validator.org}. Infine, per il PHP ed il JS, seppur non esplicitamente richiesto, sono stati sfruttati comunque sfruttati validatori: rispettivamente PhpCodeChecker\footnote{https://phpcodechecker.com} ed Esprima\footnote{http://esprima.org/demo/validate.html}. Crediamo che validare il codice, in generale, sia sempre un ottimo strumento per il debugging, la manutenzione del sito web e per i motivi precedentemente elencati.