Il primo punto per progettare un sito web accessibile è garantire una trasformazione elegante delle pagine web. Per assicurarsi ciò si è innanzitutto separato la gli aspetti riguardanti contenuto, presentazione e struttura, come si è già discusso precedentemente. Successivamente sono state fornite equivalenti testuali per ogni media, permettendo anche agli utenti non vedenti di accedere alle informazioni attraverso l'udito. Inoltre le pagine sono responsive, poiché si adattano a seconda delle dimensioni dello schermo dell'utilizzatore, grazie al vasto utilizzo di dimensioni relative nei fogli di stile e grazie all'utilizzo di punti di rottura che ne facilitano la progettazione multi-piattaforma.