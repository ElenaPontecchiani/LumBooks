Il contenuto del sito è relativamente omogeneo e risulta organizzato come una collezione di processi differenti. Per questi motivi è stato scelto uno schema organizzativo ambiguo orientato al \textit{task}. Crediamo che questa scelta sia la più appropriata in quanto vi è un numero limitato di compiti ad alta priorità, come per esempio la ricerca o l'inserimento di libri.\\
L'utente, dopo aver selezionato il \textit{task} iniziale, sarà guidato fino al raggiungimento del proprio obiettivo attraverso il riempimento di informazioni negli input opportuni e secondo una struttura sequenziale, sempre all'interno del sito. Ciò permette anche di rendere l'accesso ai contenuti il più semplice possibile, evitando possibili sovraccarichi cognitivi.