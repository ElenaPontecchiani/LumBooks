Nello sviluppo del sito sono stati perseguiti alcuni obiettivi, elencati in seguito.

\begin{itemize}
	\item \textbf{Separazione struttura-presentazione:} Questo obiettivo è senza ombra di dubbio da considerasi il più importante, in quanto il suo raggiungimento comporta una maggior facilità nel soddisfare anche gli altri obiettivi. I contenuti del sito e la loro struttura devono essere separati dalla parte di presentazione grafica: si sono quindi definite le componenti di stile nei fogli CSS, mentre il contenuto è perlopiù gestito da PHP e HTML.
	
	\item \textbf{Accessibilità: } Il sito deve poter essere fruibile agevolmente dal maggior numero utenti possibile, compresi quelli con gravi disabilità visive e/o motorie. Per raggiungere tali obiettivi, sono state adottate varie misure, fra i quali citiamo le più significative:
	\begin{itemize}
		\item TabIndex sui form;
		\item Testo alternativo per le immagini;
		\item Testi e link con buoni livelli di contrasto;
	\end{itemize}
	
	\item \textbf{Flessibilità:} Il sito deve essere consultabile da varie tipologie di dispositivi, smartphone compresi. Deve, inoltre, essere adattabile a differenti dimensioni di schermo con il minor sforzo possibile.
	
	\item \textbf{Facilità d'uso:} Per rendere il sito maggiormente intuitivo per l'utente finale si è scelto di rispettare le convenzioni del web, anteponendo, quando possibile, queste ultime all'estetica del sito. Un esempio di tale scelta si può notare nei link: essi sono di colore blu sottolineato come l'utente è solito aspettarsi, e diventato viola se già visitati. Alcuni link, posti su sfondo azzurro, sono invece bianchi per avere un livello di contrasto sufficiente.
\end{itemize}