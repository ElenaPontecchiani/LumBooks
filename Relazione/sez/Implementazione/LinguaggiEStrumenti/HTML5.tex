Il gruppo ha utilizzato il linguaggio HTML5\footnote{https://www.w3.org/TR/2017/REC-html52-20171214/} per la struttura del sito, mantenendo comunque la compatibilità con XHTML\footnote{https://www.w3.org/TR/2015/REC-xhtml-rdfa-20150317/}. Questo permette al sito di funzionare sui browser più obsoleti e migliora la comprensibilità del codice.\\
Per assicurare un codice corretto, sono state seguite le linee guida del corso di TecWeb e W3C\footnote{https://www.w3.org/standards/webdesign/htmlcss}. Il codice è stato validato utilizzando il tool di validazione W3C\footnote{https://validator.w3.org/\#validate\_by\_uri}.\\
Le regole più importanti sono:
\begin{itemize}
    \item \textbf{Chiusura tag:} ogni tag deve essere chiuso(<tag></tag oppure <tag/>);
    \item \textbf{Tag intestazione:} per ogni sezione (body, div, section) ogni tag di intestazione (h1,h2,...) deve partire sempre da h1;
    \item \textbf{Metatag:} nella sezione header, devono essere inseriti i metatag necessari per migliorare l'accessibilità verso i motori di ricerca. Questo permette al sito di avere una migliore visibilità in internet; 
    \item \textbf{Separazione struttura presentazione comportamento:} il codice HTML non deve contenere CSS o script. Questi devono essere scritti in file separati e importati nell'header;
    \item \textbf{Struttura:} il codice HTML non deve sostituire il CSS, quindi non può contenere elementi per la presentazione;\label{struttura}
    \item \textbf{Tabelle:} Le tabelle devono essere evitate quando possibile.
\end{itemize} 