Per rendere il sito più integrato nei dispositivi in cui viene utilizzato, può essere necessario usare dei setting o formati specifici, anche solo per avere una favicon universalmente riconosciuta. Ci siamo quindi avvalsi del tool \url{realfavicongenerator.net} per generare l'icona del sito, a partire dal logo in SVG. Questo ci consente di avere la sicurezza che la nostra icona sia ottimizzata in una vasta gamma di dispositivi. Oltre a ciò, il sito sopracitato genera anche dei meta tag per integrarsi in modo più stretto con vari device: ad esempio, navigando in mobile da chrome, la barra degli indirizzi assume l'accent color di LumBooks; ancora, su Android, aggiungendo uno scorciatoia al sito, esso si comporterà e verrà visto in modo simile a un'app nativa.