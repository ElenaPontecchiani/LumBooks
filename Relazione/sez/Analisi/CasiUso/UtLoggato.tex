Definiamo \textit{loggato} un utente che:
\begin{itemize}
	\item Ha eseguito con successo il login;
	\item Si è registrato con successo, ed è quindi stato rimandato alla home. In questo caso, il login è avvenuto automaticamente.
\end{itemize}

L'utente loggato eredita tutti i casi d'uso dell'utente generico, eccetto:
\begin{itemize}
	\item Registrazione;
	\item Login;
\end{itemize}

L'utente loggato dispone, inoltre, dei seguenti casi d'uso:
\begin{itemize}
	\item \hyperref[par:Inserimento]{Inserimento di un nuovo libro (\ref{par:Inserimento}});
	\item \hyperref[par:VisLibriUt]{Visualizzazione dei propri libri in vendita e venduti (\ref{par:VisLibriUt}});
	\item \hyperref[par:ModLibro]{Modifica di un proprio libro in vendita (\ref{par:ModLibro}});
	\item \hyperref[par:VendLibro]{Marcatura di un proprio libro in vendita come venduto (\ref{par:VendLibro}});
	\item \hyperref[par:RimuoLibro]{Rimozione di un proprio libro dal sito (\ref{par:RimuoLibro}});
	\item \hyperref[par:VisDati]{Visualizzazione dei propri dati di iscrizione al sito (\ref{par:VisDati}}).
\end{itemize}

\paragraph{Inserimento di un nuovo libro}\mbox{}\\
\label{par:Inserimento}
L'utente loggato entra nella pagina di inserimento libri clickando su \textit{Inserisci} nella navbar. Nella pagina di inserimento l'utente può scegliere se scegliere di inserire:
\begin{itemize}
	\item Una propria istanza di un libro già in catalogo: in questo modo, molto dei campi saranno già precompilati. Cliccando sul menù a tendina, l'utente avrà a disposizione i titoli dei volumi selezionabili.
	\item Un libro non presente in catalogo: l'utente dovrà riempire più campi rispetto alla prima opzione, ma potrà inserire un libro a libera scelta.
\end{itemize}
In caso l'utente inserisca un campo dati non conforme al formato atteso (ex. ISBN con lunghezza diversa da 13, prezzo con tre cifre decimali), sarà di ciò dal sito, e verrà invitato a inserire i dati nel modo corretto.
Se, invece, l'inserimento è avvenuto con successo, il sito informerà l'utente e lo reindirizzerà alla home.

\paragraph{Visualizzazione dei propri libri in vendita e venduti}\mbox{}\\
\label{par:VisLibriUt}
L'utente loggato entra nella pagina di visualizzazione libri clickando su \textit{I miei Libri} nella navbar. Da qui, l'utente può visualizzare i proprio libri divisi fra \textit{In vendita} e \textit{Venduti}, e rappresentati tramite delle card simili a quelle presenti nelle schermate di ricerca libro (\ref{par:RicercaLibro}), ma presentanti tre bottoni:
\begin{itemize}
	\item \hyperref[par:ModLibro]{Modifica (\ref{par:ModLibro})};
	\item \hyperref[par:VendLibro]{Venduto (\ref{par:VendLibro})};
	\item \hyperref[par:RimuoLibro]{Rimuovi (\ref{par:RimuoLibro})}.
\end{itemize}
L'utente inoltre, oltre alle informazioni che vedrebbe nella pagina di ricerca del libro, ha a dispozione la data di aggiunta del libro stesso.

\paragraph{Modifica di un proprio libro in vendita}\mbox{}\\
\label{par:ModLibro}
Nella schermata di visualizzazione dei propri libri, l'utente preme il tasto \textit{Modifica} sulla card di un libro in vendita. Viene quindi caricata la pagina di modifica del libro, molto simile a quella di inserimento. L'utente, quindi, modifica i campi a sua scelta, e poi conferma i cambiamenti apportarti. Se tutto è andato bene, il sito reindirizza l'utente ai propri libri; in caso contrario, l'utente viene informato dell'errore e invitato a riprovare.

\paragraph{Marcatura di un proprio libro in vendita come venduto}\mbox{}\\
\label{par:VendLibro}
Nella schermata di visualizzazione dei propri libri, l'utente preme il tasto \textit{Venduto} sulla card di un libro in vendita. Ciò comporta lo spostamento del libro nella sezione \textit{Libri Venduti} tramite un refresh della pagina. Un libro che viene segnato come venduto non può essere rimesso in vendita, e rimangono disponibili solo le seguenti operazioni:
\begin{itemize}
	\item Visualizzazione del libro (\ref{par:SpecLibro});
	\item \hyperref[par:RimuoLibro]{Rimozione del libro (\ref{par:RimuoLibro})}.
\end{itemize}

\paragraph{Rimozione di un proprio libro dal sito}\mbox{}\\
\label{par:RimuoLibro}
Nella schermata di visualizzazione dei propri libri, l'utente preme il tasto \textit{Rimuovi} sulla card di un libro. Il libro viene rimosso dal sito tramite un refresh della pagina, e non è più possibile reperirlo nella piattaforma.

\paragraph{Visualizzazione dei propri dati di iscrizione al sito}\mbox{}\\
\label{par:VisDati}
L'utente, una volta loggato, visualizza la propria immagine profilo nella navabar. Se invece, al momemento dell'iscrizione, l'utente avesse scelto di non caricarne una, visualizzerà un'immagine generica.
Per visualizzare il resto dei suo dati, l'utente clicca il link \textit{I miei dati} sulla navbar, e viene portato a una pagina riassuntiva con le sue informazioni.







