Definiamo \textit{generico} un utente che:
\begin{itemize}
	\item visita il sito senza possedere un account Lumbooks;
	\item visita il sito senza effettuare la login;
\end{itemize}

L'utente generico dispone dei seguenti casi d'uso:
\begin{itemize}
	\item \hyperref[par:VisHome]{Visualizzazione Homepage del sito (\ref{par:VisHome}});
	\item \hyperref[par:VisLibriVendita]{Visualizzazione dei libri correntemente in vendita nel sito (\ref{par:VisLibriVendita}});
	\item \hyperref[par:VisLibriCatalogo]{Visualizzazione dei libri appartenenti al catalogo del sito(\ref{par:VisLibriCatalogo}});
	\item \hyperref[par:RicercaLibro]{Ricerca di uno specifico libro tra quelli in vendita (\ref{par:RicercaLibro}});
	\item \hyperref[par:SpecLibro]{Visualizzazione di un particolare libro in vendita (\ref{par:SpecLibro}});
	\item \hyperref[par:VisAbout]{Visualizzazione pagina "Chi siamo" (\ref{par:VisAbout}});	
	\item \hyperref[par:Reg]{Registrazione (\ref{par:Reg}});
	\item \hyperref[par:Login]{Login (\ref{par:Login}}).
\end{itemize}

\paragraph{Visualizzazione Homepage del sito}\mbox{}\\
\label{par:VisHome}
L'utente generico può entrare nella Homepage in diversi modi:
\begin{itemize}
	\item Se è appena entrato nel sito è la prima pagina che viene visualizzata;
	\item Se si trova in un'altra pagina, può raggiungere la Homepage cliccando la sezione "Home" nella navbar;
	\item Se si trova in un'altra pagina, può raggiungere la Homepage cliccando sopra la scritta "Lumbooks" presente nell'header;
\end{itemize}
All'interno di questa pagina l'utente può apprendere le principali funzionalità del sito e visionare le ultime offerte inserite inserite (nell'apposita sezione in basso "Le nostre ultime offerte!")



\paragraph{Visualizzazione dei libri correntemente in vendita}\mbox{}\\
\label{par:VisLibriVendita}
L'utente generico entra nella pagina di visualizzazione dei libri in vendita cliccando la sezione "in vendita" presente nella navbar".\\
I libri visualizzati in questa sezione possono essere appartenenti al catalogo Lumbooks o esterni a esso. A seconda della tipologia di libro vi sono delle caratteristiche obbligatorie e altre opzionali. Ogni libro quindi è visualizzato all'interno di una scheda, che ne contiene le seguenti caratteristiche (a prescindere che si tratti di un libro appartenete al catalogo o meno):
\begin{itemize}
	\item Titolo (obbligatorio, sempre presente);
	\item Autore (obbligatorio, sempre presente);
	\item Prezzo (obbligatorio, sempre presente);
	\item ISBN (presente opzionalmente);
	\item Foto del libro (presente opzionalmente);
\end{itemize}

\paragraph{Visualizzazione di un particolare libro in vendita}\mbox{}\\
\label{par:SpecLibro}
Nel caso l'utente volesse ulteriori informazioni su un determinato libro, il titolo del libro presente nella scheda è un link che indirizza a un'altra pagina, dove sono visualizzati in maniera più specifica le caratteristiche del libro. Tra le caratteristiche aggiuntive in questa pagina possiamo trovare:
\begin{itemize}
	\item Editore;
	\item Che tipo di materiale di studio è (libro, dispense..);
	\item Anno di pubblicazione;
	\item Edizione;
	\item Casa editrice;
	\item Una breve descrizione del libro;
	\item Recapiti del venditore.
\end{itemize}

\paragraph{Visualizzazione dei libri del catalogo}\mbox{}\\
\label{par:VisLibriCatalogo}
L'utente generico entra nella pagina di visualizzazione dei libri appartenenti al catalogo LumBooks cliccando la sezione "catalogo" presente nella navbar".\\
All'interno della pagina sono visualizzati i libri del catalogo, all'interno di schede analoghe a quelle riportate nella pagina "In vendita".\\
Le caratteristiche qui riportate per ogni libro sono:
\begin{itemize}
	\item Titolo;
	\item Autore;
	\item Casa editrice;
	\item Corso a cui fanno riferimento;
\end{itemize}
Anche in questa pagina il titolo del libro presente in ogni scheda è un link: cliccandolo si viene reindirizzati alla pagina "risultati della ricerca", che visualizza tutti i libri in vendita corrispondenti al libro di catalogo di cui si è selezionato il titolo. Per esempio se clicco il titolo del libro "Programmazione: semplice è bello" vengo reindirizzato alla pagina di risultati che visualizza tutte le offerte di questo libro presenti nel sito.

\paragraph{Ricerca di un libro tra quelli in vendita}\mbox{}\\
\label{par:RicercaLibro}
L'utente generico entra nella pagina di ricerca dei libri in vendita cliccando la sezione "Cerca un libro" presente nella navbar".\\
La pagina offre all'utente un motore di ricerca statico che permette di ricercare determinati materiali all'interno del sito.\\
I parametri per la ricerca sono:
\begin{itemize}
	\item Titolo;
	\item Autore;
	\item Corso;
	\item Editore;
	\item ISBN;
	\item Parola chiave;
\end{itemize}
L'utente può decidere quali campi compilare a seconda del grado di precisione della ricerca che vuole effettuare: nessun campo è a compilazione obbligatoria. Inoltre la compilazione dei campi non è case sensitive.\\
Una volta compilati i campi l'utente preme il tasto "Cerca" ed effettua la ricerca. Gli esiti possibili sono due:
\begin{itemize}
	\item La ricerca è andata a buon fine, quindi viene visualizzata una pagina contenente tutti i risultati corrispondenti a tale ricerca. La modalità di visualizzazione dei risultati è analoga a quella presente nella sezione "in vendita": ogni libro è contenuto in una scheda che ne descrive le caratteristiche.;
	\item La ricerca non sortisce risultati: non esistono corrispondenze per la ricerca effettuata: viene visualizzato un segnale di errore che invita a eseguire un'altra ricerca;
	
\end{itemize}

\paragraph{Visualizzazione pagina "Chi siamo"}\mbox{}\\
\label{par:VisAbout}
L'utente generico entra nella pagina di visualizzazione dei libri in vendita cliccando la sezione "chi siamo" presente nella navbar".\\
All'interno di questa pagina statica sono contenute le informazioni riguardanti i componenti del gruppo e la nascita del progetto Lumbooks.

\paragraph{Registrazione}\mbox{}\\
\label{par:Reg}
L'utente generico entra nella pagina di registrazione cliccando la sezione "Registrati" presente nell'header".\\
La pagina offre all'utente generico la possibilità di creare il proprio account personale LumBooks. Un utente può possedere più account LumBooks, a patto che vengano registrati con email diverse.\\
Per effettuare la registrazione l'utente deve inserire una serie di dati, alcuni obbligatori (quelli contrassegnati con *) e altri opzionali. al termine dell'inserimento l'utente preme il tasto registrati, e gli esiti possibili sono due:
\begin{itemize}
	\item La registrazione è andata a buon fine, l'inserimento dei dati è avvenuto in maniera corretta. Viene visualizzato un messaggio che indica che l'operazione si è conclusa correttamente e si viene reindirizzati alla homepage ;
	\item La registrazione non è andata a buon fine, e questo può essere dovuto a diverse motivazioni. Se i dati inseriti non sono corretti (per esempio email nel formato sbagliato, ripetizione della password non coincide) vengono visualizzati dei messaggi di errore che indicano il/i problemi e in seguito si invita a reinserire i dati sbagliati (i dati inseriti correttamente rimangono nei campi di compilazione);
\end{itemize}

\paragraph{Login}\mbox{}\\
\label{par:Login}
L'utente generico entra nella pagina di login cliccando la sezione "Accedi" presente nell'header".\\
La pagina offre all'utente già in possesso di un account Lumbooks la possibilità di accedervi, inserendo le proprie credenziali (email e password). Gli esiti dell'inserimento possono essere due:
\begin{itemize}
	\item L'inserimento dei dati è corretto; l'utente accede correttamente al suo account Lumbooks. Al termine dell'inserimento viene reindirizzato alla Homepage;
	\item L'inserimento dei dati non è corretto; l'utente non accede al suo account Lumbooks. Possono essere diverse le casistiche che portano a tale risultato: password e/o email non corretta. In ogni caso viene indicato all'utente qual'è il problema, attraverso appositi messaggi di errore. 
\end{itemize}
L'inserimento inoltre contiene un controllo per quanto riguarda l'inserimento dell'email: viene controllato che il formato sia effettivamente quello di un email, e in caso contrario viene segnalato all'utente.



