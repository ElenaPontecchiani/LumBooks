Il progetto LumBooks si propone come mezzo intermediario all'interno di in una compravendita di materiale universitario tra gli studenti; pertanto quest'ultimi saranno l'utenza finale primaria del sito, o comunque soggetti legati all'ambito accademico. \\ 
Nel particolare inizialmente la piattaforma viene pensata per essere destinata a studenti frequentanti il corso di studi di Informatica presso l'università di Padova (tant'è che il catalogo contiene prettamente libri appartenenti a questo corso), ma in futuro il sito potrebbe espandersi e rivolgersi a qualsivoglia studente frequentante una qualsiasi università.\\
In prospettiva di una futura crescita, quindi, l'utente finale potrebbe non essere specializzato o pratico di strumenti informatici, dal momento che potrebbe provenire da diverse facoltà. Per questo è necessario adottare un linguaggio informale, semplice e di facile intuizione, che possa essere compreso dalla maggior parte delle persone. Lo stesso discorso vale per il layout e la struttura del sito, che si prefiggono di essere intuitivi e veloci, rispettando le principali convenzioni del web.\\
