Il progetto \emph{LumBooks}, svolto come finalità del corso di Tecnologie Web nell'anno accademico 2018-2019,
 si propone di implementare un sito Internet per la compravendita di materiale universitario tra studenti.\\
 Il nome "LumBooks", infatti, nasce dall'insieme delle parole Lum (che fa riferimento all'aula Lum250 del complesso Paolotti, dove si svolgono le lezioni del corso di studio di Informatca) e books.\\
Per conseguire lo scopo stabilito, la natura del sito è fortemente interattiva. Per i visitatori del sito è possibile infatti registrarsi e creare il proprio account personale, grazie al quale l'utente può cercare e vendere il materiale di studio che desidera, valutando offerte di altri utenti del sito.\\
Nonostante ciò la registrazione al sito non è obbligatoria, e nel caso un visitatore non volesse iscriversi il sito ha in qualsiasi caso funzionalità informative, in quanto permette comunque al visitatore esterno di cercare un determinato libro visualizzando le offerte che il sito propone.\\

Il sito è stato sviluppato con l’intenzione di essere poi pubblicato in internet, dunque si
è data molta importanza alla sua usabilità, rispettando gli standard W3C, la separazione tra struttura, presentazione, comportamento e le regole di accessibilità richieste.